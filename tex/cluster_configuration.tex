\section{Configuración del Cluster}

Para empezar con la configuración del cluster iniciamos instalando unos paquetes básicos, esto
usando los archivos de \emph{Ansible} que teníamos, y con esto se instalaron paquetes básicos como
editores de texto (\emph{Vim} o \emph{Emacs}), drivers (como por ejemplo los de \emph{Infiniband}) y 
\emph{ntp}, además se instalaron los compiladores de \emph{Intel}, \emph{NFS}, \emph{NFSoRDMA}
(aunque se estudiaba la posibilidad de usar \emph{GlutterFS}), también \emph{SLURM}, \emph{QUEST} y 
por último \emph{PRESTO}, ahora bien, para poder instalar \emph{PRESTO} primero debiamos tener
algunas dependencias instaladas como \emph{glib} (versión 2.66.4), \emph{fftw} (versión 3.3.9), 
\emph{cfitsio} (versión 3.49), \emph{tempo} (versión 1.0.0) y \emph{pgplot} (versión 5.2).